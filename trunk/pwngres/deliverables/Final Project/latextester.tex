\documentclass{beamer}

\usetheme{Copenhagen}

\title{test}

\begin{document}

\begin{frame}
	\maketitle
\end{frame}


\begin{frame}
	\tableofcontents
\end{frame}

%\subsection{Urban Search and Rescue}
%\section{Selection of an Algorithm}
%\subsection{MDS Robot}
%\subsection{Previous Work}

\begin{frame}
	\section{section one}
	\frametitle{wtf}
	frame A
\end{frame}

\begin{frame}
	\frametitle{Mobile Robotics}
	frame B
\end{frame}

\section{section two}
%\frame
%{
%	\frametitle{What we need?}
%	
%	A computer vision algorithm
%	
%	\begin{itemize}
%		\item To identify human bodies
%		\item To identify other objects		
%	\end{itemize}
%	
%	Take into account:
%	\begin{itemize}
%		\item The characteristics of the disaster area scenario
%		\item The hardware capabilities of our robotic platform
%	\end{itemize}	
%}

%
%\frame
%{
%	\section{Previous Work}
%	\frametitle{Computer vision algorithms}
%	
%	Consider two options:
%	
%	\begin{itemize}
%		\item Algorithms that use vision data only (e.g. camera images)
%		\item Algorithms that use vision data integrated with data from another sensor 
%				(e.g. laser readings)	
%	\end{itemize}
%	
%}

%\section{Robotic Platform}
%\frame
%{
%	\frametitle{MDS Robot}
%	
%	\begin{minipage}[t]{.47\textwidth}
%		\begin{itemize}
%		\item Personal Robots Group, MIT Media Lab
%		\item Newest robotic platform
%		\item Mobile-Dexterous-Social
%		\item Two sensors of interest:
%				\begin{itemize}
%					\item Stereo camera pair
%					\item Laser rangefinder
%				\end{itemize}
%		\end{itemize}
%	\end{minipage}
%	\hfill
%	\begin{minipage}[t]{.47\textwidth}
%		\begin{figure}
%			\centering
%			\includegraphics[width=4cm]{mdspic.jpg}
%			\caption{MDS \cite{mdsweb}}
%	     \end{figure}	
%	\end{minipage}
%}

%\frame
%{
%	\frametitle{References}
%	
%	\small{
%	\begin{thebibliography}{Douillard et al., 2007}

%
%		\bibitem{mdsweb} 
%				Personal Robots Group
%				\newblock \href{http://robotic.media.mit.edu/projects/robots/mds/social/social.html}{http://robotic.media.mit.edu/projects/robots/mds/social/social.html}

%		\bibitem[Douillard et al., 2007]{douillard}
%			    B. Douillard, D. Fox, and F. Ramos. 
%			    \newblock A spatio-temporal probabilistic model for multi-sensor multi-class 
%			    				object recognition. 
%				\newblock In \textit{Proceedings of the International Symposium of Robotics 
%			    				Research (ISRR)}, 2007.
%			    
%	 	\bibitem[Felzenszwalb et al., 2008]{felzenszwalb}
%				P. Felzenszwalb, D. McAllester, D. Ramaman.  
%				\newblock A Discriminatively Trained, Multiscale, Deformable Part Model.  
%				\newblock In \textit{Proceedings of the IEEE CVPR}, 2008.
%      			
%	\end{thebibliography}
%	}
%}

%\frame
%{
%	\frametitle{Include a picture in your slideshow}
%	
%	\begin{figure}
%	\centering
%	%\includegraphics[width=5cm]{sample_rate.png}
%	\caption{Our own test of downsampling}
%	\label{fig:test_down_sampling}
%	\end{figure}
%}

%\href{http://robotic.media.mit.edu/projects/robots/mds/social/social.html}{http://robotic.media.mit.edu/projects/robots/mds/social/social.html}.


%\frame
%{
%	\frametitle{Include a table in your slideshow}
%	
%	\begin{table}
%	\centering
%	\begin{tabular}{|c|c|c|} \hline \hline
%	Restrained     & Swivel         & Telemetry      \\ \hline \hline
%	$\sim$ 400bpm  & $\sim$ 380bpm  & $\sim$ 310 bpm \\
%	$\sim$ 140mmHg & $\sim$ 120mmHg & $\sim$ 100mmHg \\ \hline \hline
%	\end{tabular}
%	\caption{Heart beat and blood pressure using different monitoring methods}
%	\label{tbl:kramer}
%	\end{table}
%}





\end{document}
